\section{The Land Surface overview}\label{introduction}

The Earth System behaves as a single, self-regulating system comprised of physical, chemical, biological, and human components \citep{Pronk2002}. In order to understand the interactions between those components, natural scientists have done a lot of work on creating and improving Earth\textquotesingle s system models. Even though humanity has always been under the influence of the weather for several reasons, e.g., agriculture, natural catastrophes, among others, until the 19$^{th}$ century weather forecast was based on empirical rules with limited understanding of physical mechanisms. The advent of new theories based on preexisting laws of mass continuity, conservation of momentum, the first and second laws of thermodynamics allowed the prediction of the state of the atmosphere in the future through numerical methods \citep{Lynch2008}.

The regional mathematical models of weather forecast were sooner extended to the globe, in order to evaluate the behavior of the atmosphere as a whole. The first type of general circulation models could realistically depict patterns in the troposphere, however it was the appearing of new knowledge related to other areas of the Earth\textquotesingle s system as the oceans, sea ice, soil, and vegetation, and the concomitant increasing computational power that led the scientific community to the development of more realistic coupled models, the so-called global climate models (GCMs).

GCMs are currently used for understanding the present and predict the future climate. Furthermore, submodels interlinked with GCMs allow researchers to understand and predict the interaction between climate and ecosystems, which are directly related to food production, plant, and animal species distribution on planet Earth, and ultimately has an impact on human life itself. GCMs require the fluxes of radiation, heat, water vapour, and momentum across the land-atmosphere interface to be specified. These fluxes are calculated by models called land surface models (LSMs). 

LSMs are an important tool for understanding land-surface-atmosphere dynamics and interactions, and climate-carbon feedbacks \citep{loew2014}. These models have evolved from simple, unrealistic schemes into credible representations of the global soil-vegetation-atmosphere transfer system, as advances in plant physiological and hydrological research, advances in satellite data interpretation, and the results of largescale field experiments have been exploited \citep{Sellers1997}.

\subsection{The surface energy balance}

\subsection{The carbon cycle}

\section{The Land Surface models}

\subsection{Background}

\subsection{First generation models: bucket scheme}

\subsection{Second generation models: two layers scheme}

\subsection{Third generation models: `greening'}

\section{Impacts of vegetation architecture}

\subsection{Turbulent flow}

Seidl, R., Rammer, W., & Blennow, K. (2014). Simulating wind disturbance impacts on forest landscapes: Tree-level heterogeneity matters. Environmental Modelling & Software, 51, 111. https://doi.org/10.1016/j.envsoft.2013.09.018

\subsection{Radiation partitioning}

\section{Direction of the thesis}

